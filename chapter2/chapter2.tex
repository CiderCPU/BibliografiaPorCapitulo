\providecommand{\main}{..} % makes sure that the directory reference is relative to the subfile
\documentclass[../main.tex]{subfiles}

\begin{document}
    \newrefsection
    \definecolor{chapcolor}{RGB}{11,175,230} % chapter 2 color
    \chapter{Second Chapter}
    Hello there. This is a test.
    \begin{equation}
        \label{eq:chap2:1}
        n + m = 2q
    \end{equation}
    La expresión anterior es conocida como una \hlt{ecuación}. \textquestiondown Ves el número a la
    derecha de la ecuación? Ese número sirve como etiqueta para hacer referencia a esa expresión. En
    vez de escribir ``\(n + m = 2q\)'' repetidas veces, podemos simplemente escribir ``la ecuación \ref{eq:chap2:1}''
    para hacer referencia a esa expresión, sin necesidad de repetirla una y otra vez.
    
    Como parte de las funcionalidades de este PDF, las etiquetas en color {\color{octyred}rojo} funcionan 
    como hipervínculos: hacer click sobre una de estas etiquetas te llevará a la definición, teorema, 
    expresión, etc. a la que hacen referencia. \textexclamdown Prueba esta funcionalidad visitando la 
    ecuación \ref{eq:chap1:1}, del capítulo anterior! \cite{aluffi09}
    \printbibliography
\end{document}